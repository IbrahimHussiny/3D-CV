\documentclass[a4paper, twoside, english]{article}

\usepackage{amsmath}
\usepackage{amsfonts}
\usepackage{ihci}
\usepackage{graphicx}
\usepackage{subfig}

\graphicspath{{./../figures/}}

\title{Exercise 3 - Theory}
\author{
	Abdelaziz, Ibrahim
	\and
	Somkiadcharoen, Robroo
	\and
	Berg, Oliver
}
\date{\today}

\begin{document}
\maketitle


\section{Theory}

\subsection{T1}
Point $x_0$ constrains the position of $x_1$ to lie on the corresponding epipolar line $l_1$ on the second image. where $l_1 = Fx_0$.

\subsection{T2}
Using the equation $x_i = K_i[R_i|t_i].X$ we can get two linearly independent equations for each camera, for two cameras we will have 4 linearly independent equations to solve for 3 unkowns in $X$.

\subsection{T3}
Since epipole $e$ is the image of the center of the other camera $C'$ it can be computed as $e = PC'$. 
Since all epipolar planes rotate around the same baseline, then all epipolar lines intersect at the epipole point.

\subsection{T4}
For each corresponding points $x$ , $x'$ it satfies the following equation $x'^TFx=0$, where F is a 3*3 matrix (9 unkowns). We can set up a homogeneous linear system with 9 unkowns, and given enough corresponding points we can solve for the 9 unkowns of F.
\subsection{T5}
Given the camera calibration, the fundamental matrix can be computed using the following equation \begin{equation} F = K'^-^T.[t].R.K^-^1 \end{equation}
\section{Implementation}

You may find the implementation code inside the \lstinline{main.py} file. To run the code, call \lstinline{python main.py}.

\end{document}
