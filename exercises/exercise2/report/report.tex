\documentclass[a4paper, twoside, english]{article}

\usepackage{amsmath}
\usepackage{amsfonts}
\usepackage{ihci}
\usepackage{graphicx}
\usepackage{subfig}

\graphicspath{{./../figures/}}

\title{Exercise 2 - Theory}
\author{
	Abdelaziz, Ibrahim
	\and
	Somkiadcharoen, Robroo
	\and
	Berg, Oliver
}
\date{\today}

\begin{document}
\maketitle


\section{Theory}

\subsection{Homography Definition}
In case of $P^2$ (2D Projection plane) we have homogeneous coordinates as $[x_1, x_2, x_3]$ and the $H$ transformation matrix is of size $3\times3$ as $\begin{bmatrix}
h_11&h12&h13 \\
h_21&h22&h23 \\
h_31&h32&h33 \\
\end{bmatrix}$ where 8 of them are independent ratios(DOF) and another one is the gain.

Using the same logic as above, you can get a point from $P^n$ (n-dimensional Projection Space) as $[x_1, x_2, ... x_n+1]$ and the $H$ transformation matrix is of size $(n+1)\times (n+1)$. Thus, $(n+1)^2-1$ DOF. Motivated From \cite{Stackoverflow}\cite{DynamicPntoPnAlgnmnt}

\subsection{Line preservation}

Given that a point $x=[x1,x2,x3]$ is a point in 2D Projection plane which is also on a line $l$, and all the points are on $l$ which gives $l^Tx_i=0$. We can derive
\begin{equation}
l^Tx_i=0=l^TH^{-1}Hx_i\label{eq:1}
\end{equation}
From \eqref{eq:1} we get that the points $x'=Hx_i$ that is transformed lie on the line $l'=l^TH^{-1}$ In other words, we can perceived from the equation that points are transformed by $x'$ and line is transformed by $l'$

Highly Motivated by \cite{ProjectiveGeomUMD} \cite{ProjectiveGeomIIT}


\bibliographystyle{IEEEtran}
\bibliography{bibliography.bib}
\end{document}