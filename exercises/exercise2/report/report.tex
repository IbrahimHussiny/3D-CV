\documentclass[a4paper, twoside, english]{article}

\usepackage{amsmath}
\usepackage{amsfonts}
\usepackage{ihci}
\usepackage{graphicx}
\usepackage{subfig}

\graphicspath{{./../figures/}}

\title{Exercise 1 - Theory}
\author{
	Abdelaziz, Ibrahim
	\and
	Robroo
	\and
	Berg, Oliver
}
\date{\today}

\begin{document}
\maketitle

\section{Properties of Rotation Matrices}

\subsection{Showing that $U^T = U^{-1}$ holds for \textit{general} rotation matrix $U \in \mathbb{R}^3$}

Generally note that $UU^{-1}=I$ subject to $U^{-1} = U^T$, as provided by $U$ being rotation matrix and as such orthogonal, yields $UU^T = I$.

For given column-vectors $c_1, c_2, c_3$ of $U$ we observe 

\begin{equation*}
	<c_i, c_j> = \delta_{i j}
\end{equation*}

which in combination with

\begin{equation*}
	U^T = (c_1, c_2, c_3)^T = (c_1^T, c_2^T, c_2^T)
\end{equation*}

results in

\begin{equation*}
	UU^T = (<c_i, c_j>)_{1 \le i, j \le 3} = I
\end{equation*}

Motivated by \cite{MathematicsSEMatrixTransposeIdentity}\cite{MathematicsSEMatrixTransposeIdentity2}\cite{WikiOrthogonaleMatrix}.


\subsection{Geometric interpretation of determinant of 3x3 matrix}

Bla

\section{Transformation Chain}

Blub

\bibliographystyle{IEEEtran}
\bibliography{bibliography.bib}
\end{document}